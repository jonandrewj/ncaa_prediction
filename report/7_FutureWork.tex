\section{Future Work}

\subsection{Probabilistic Upsets}

Because we were unsuccessful in improving the model beyond the accuracy achieved using the seeding data we collected, we decided to explore the possibility of predicting the most likely upsets. 

In order to do so, we decided to define upsets as games in which the seeding between both teams differed by 5 or more seeds. 
The intention of this sub setting was to focus on the features that might best predict upsets. 
Additionally, upsets are a relatively rare occurrence in these games. 
In fact, in games where the seed differed by at least five games, upsets occurred with a frequency of roughly 21\%.
Because the dataset was skewed towards few upsets, the data was undersampled to create a dataset with 50\% upsets and 50\% non-upset games. 
Again, this was done in hopes of creating a dataset that is more likely to find attributes that are predictive of upset games. 

Using this data, a logistical regression model was fit to predict upsets. 
The predictive accuracy on a test set from original data yielded an accuracy of roughly 73\%. 
Although this accuracy is not very high, the logistic regression model allowed for a probabilistic output. 
In other words, we were able to get a list of the most likely upsets probability wise. 

These probabilities offer the potential of adding some picks for upsets to one’s bracket that our other model would not have chosen. 
Because bracket prediction is so difficult, it is worthwhile to add some potential upsets to one’s bracket selected by this model. 
By adding these upsets to one’s bracket, the chances of winning a bracket challenge will increase. 

This also may be the subject of future work, as the real challenge in creating good brackets is to figure out which upsets are most likely. 
In the future it would certainly be worthwhile to try more models and different techniques of finding the most likely upsets in the tournament.

\subsection{Bracket Creation}

Our next step from these small tests would be to try to create brackets that were randomly generated using our upset predictor and see if we can randomly pick the correct upsets.
The mean idea is that predicting the most likely victor for all the games, even when 100\% correct, won't produce a bracket that wins in any of the competitions.
However, we would be interested to see if maybe a set of 100 brackets, each with an intelligent set of upsets, would have 1 that turns out to be a really good bracket.
