\documentclass[conference]{IEEEtran}
% Use to allow footnotes in the title.
%\IEEEoverridecommandlockouts

\ifCLASSOPTIONcompsoc
  % IEEE Computer Society needs nocompress option
  % requires cite.sty v4.0 or later (November 2003)
  \usepackage[nocompress]{cite}
\else
  % normal IEEE
  \usepackage{cite}
\fi

\usepackage{booktabs}
\usepackage{calc}
\usepackage{enumitem}
\usepackage{graphicx}
\usepackage{microtype}
\usepackage{tabularx}
\usepackage{url}
\usepackage[table]{xcolor}
\usepackage{xspace}

\newcommand*\rot{\rotatebox{90}}

\newcounter{tempstatlab}
\newcounter{tempstatlong}

\hyphenpenalty 10000
\exhyphenpenalty 10000

% To-do boxes
\newcommand*{\todo}[1]{{\color{red}\bf  TODO: #1}}

\begin{document}

\title{NCAA Men's Basketball Tournament Prediction}

\author{
  \textbf{Andrew Jacobson, Nate Jenkins, Trevor Smith, Joshua Stephens}\\
  CS 478, Winter 2018\\
  Department of Computer Science\\
  Brigham Young University\\
}
\maketitle

% comment this out for accepted papers - inserts page numbers for submission
\thispagestyle{plain}
\pagestyle{plain}

%% Page limite EuroUSEC - 10 pages body, 20 pages max

\begin{abstract}
Predicting the outcomes of the NCAA Men's Basketball Tournament has long fascinated many individuals.
We applied numerous machine learning models to the task, using season statistics and prior tournament data to predict future winners.
We find that a team's seeding dominates the decision making process.
Further features from a team's season statistics can be used to slightly improve prediction accuracy.
Upsets are particularly also particularly difficult for our chosen machine learning algorithms to predict.
\end{abstract}

\section{Introduction}
\todo{Write Introduction}
\section{Methodology}

\subsection{Data Collection}
We collected our data from an online database on Kaggle\footnote{https://www.kaggle.com/c/mens-machine-learning-competition-2018/data}
This dataset provided season statistics for each NCAA basketball team since 2003.
We also acquired expert rankings from ESPN\footnote{http://www.espn.com/mens-college-basketball/rpi} and Kaggle.

This raw data was not organized in a way that allowed for accurate training models.
This was primarily due to team stats and records being organized by winning and losing team.
This caused the learning model to always select the winning team, since the winning teams stats were always input in the same places.
To rectify this, we had to randomly shuffle the wininng and losing teams into teams labled A and B and then classify the output as either A or B winning.

From this processed data, we used a number of combinations to see what different data sets would allow for prediction.
We primarily focused on three groups of data. 
First, we wanted to focus on exclusively the seeding as a baseline accuracy that we would strive to beat.
Next, we looked at several of the prominent ranking algorithms such as RPI and KenPom to see how well a community of ranking algorithms could do.
Then, we looked at a set of season statistics that we organized for each team. 
Finally, we combined these three datasets together to take advantage of all of our data.

\subsection{Learning Process}
We wanted to be able to effectively use all of the techniques that we had used in class to train and test our data.
A nice solution for this was to use the Python library sklearn which provides a nice high level abstraction.
With this library, we were able to structure our training to include all of the following supervised learning models: logistic regression, decision tree, naive bayes, neural network, random forest, boosted forest, gaussian process, k-nearest neighbor, support vector machine, and several ensembles.
We were able to structure our code to see results from all of these learners and make educated decisions on which models consistently performed the best and how we could then use ensembles of those models to further improve the results.
\section{Initial Results}
\todo{Write about Initial Results}
\section{Feature Enchancement}
\todo{Write about adapting the features.}
\section{Final Results}
\todo{Write about final results}
\section{Discussion}
\subsection{Importance of Seed}
\todo{Discuss the importance of the seeding data}
\subsection{Importance of Season Statistics}
\todo{Discuss the importance of the season statistics}
\section{Future Work}

\subsection{Probabilistic Upsets}

Because we were unsuccessful in improving the model beyond the accuracy achieved using the seeding data we collected, we decided to explore the possibility of predicting the most likely upsets. 

In order to do so, we decided to define upsets as games in which the seeding between both teams differed by 5 or more seeds. 
The intention of this sub setting was to focus on the features that might best predict upsets. 
Additionally, upsets are a relatively rare occurrence in these games. 
In fact, in games where the seed differed by at least five games, upsets occurred with a frequency of roughly 21\%.
Because the dataset was skewed towards few upsets, the data was undersampled to create a dataset with 50\% upsets and 50\% non-upset games. 
Again, this was done in hopes of creating a dataset that is more likely to find attributes that are predictive of upset games. 

Using this data, a logistical regression model was fit to predict upsets. 
The predictive accuracy on a test set from original data yielded an accuracy of roughly 73\%. 
Although this accuracy is not very high, the logistic regression model allowed for a probabilistic output. 
In other words, we were able to get a list of the most likely upsets probability wise. 

These probabilities offer the potential of adding some picks for upsets to one’s bracket that our other model would not have chosen. 
Because bracket prediction is so difficult, it is worthwhile to add some potential upsets to one’s bracket selected by this model. 
By adding these upsets to one’s bracket, the chances of winning a bracket challenge will increase. 

This also may be the subject of future work, as the real challenge in creating good brackets is to figure out which upsets are most likely. 
In the future it would certainly be worthwhile to try more models and different techniques of finding the most likely upsets in the tournament.

\subsection{Bracket Creation}

Our next step from these small tests would be to try to create brackets that were randomly generated using our upset predictor and see if we can randomly pick the correct upsets.
The mean idea is that predicting the most likely victor for all the games, even when 100\% correct, won't produce a bracket that wins in any of the competitions.
However, we would be interested to see if maybe a set of 100 brackets, each with an intelligent set of upsets, would have 1 that turns out to be a really good bracket.

\section{Conclusion}
Predicting a perfect bracket, given the 9.2 quadrillion permutations, remains difficult, even with the aid of modern machine learning models.
To reiterate our findings included:
\begin{enumerate}
	\item Baseline prediction using only seeding data and choosing the higher seed to win generally yields around 70\% accuracy.
	\item Improved predictions can be made by using team statistics and rankings, though these improvements are generally only by a few percentage points.
	\item If a perfect bracket is to be found, the upsets must be inserted and can be better guessed using probabilistic methods.
\end{enumerate}

While the perfect bracket may remain a guessing game, producing a set of strong brackets to compete against friends, family, and colleagues with is certainly obtainable.



\bibliographystyle{IEEEtran}
\bibliography{main}
\section{References}
\begin{enumerate}
	\item Season Data. Retrieved February 20, 2018, from https://www.kaggle.com/c/mens-machine-learning-competition-2018/data
	\item Team Rankings. Retrieved February 20, 2018, from http://www.espn.com/mens-college-basketball/rpi
\end{enumerate}

\end{document}
