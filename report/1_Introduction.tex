\section{Introduction}

Predicting the outcome of sporting events has been an interesting topic for some time.
People's interest in such events is manifest in activities like sports betting and fantasy teams.
One of the most popular prediction games is creating brackets for the NCAA Men's Basketball Tournament, often referred to as March Madness.

This particular tournament invites the believed 64 best teams to compete to determine the season's final champion.
These 64 teams play a combined total of 63 games.
This yields a total of over 9.2 Quadrillion possible brackets.
The odds of selecting a perfect bracket are so small that every year Warren Buffett promises to award any individual to do so \$1,000,000 per year for the rest of their life.

This extremely large space means that some sort of reasoning must be used to select a competitive bracket.
Human intuition usually has bias towards certain aspects of the selection, such as selecting  favorite schools to win.

Our project wanted to see what accuracy a machine is capable of when selecting a bracket.
To do this, we used a variety  of machine learning algorithms in combination with season and tournament statistics since 2003.

To quickly summarize our results:
\begin{enumerate}
	\item \textbf{Team Seeding is Important}: Due to the structure of seeded tournaments, better teams should win with much more regularity.
	\item \textbf{Season Statistics have an Minor Impact}: Including raw season statistics for individual teams can help improve accuracy, however, these statistics are the basis for seeding and typically only provide minimal improvements.
\end{enumerate}